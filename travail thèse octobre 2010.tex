\documentclass[a4paper,french,12pt]{book}
\usepackage[T1]{fontenc}
\usepackage [frenchb]{babel}
\usepackage[utf8]{inputenc}
\textwidth19cm
\oddsidemargin-.5in
\evensidemargin-.5in
\topskip1cm
\footskip1cm
\begin{document}
\title{compilation de documents pour la partie empirique de la thèse}
\part{burnout}
\chapter{concept du burnout}

\section{livre: burnout et traumatismes psychologiques (Boudoukha A.H., 2009. chap1: le burnout. pp:1--28)}
Le burnout a été introduit par Bradley en 1969 pour qualifier les individus qui présentent un stress particulier et massif en raison de leur travail. Il a ensuite été repris par Freudenberger en 1974 et Maslach en 1976. Le travail de cette dernière a permit de l’identifier chez les infirmières, les médecins, les enseignants et les assistants sociaux. En général, il concerne les personnes dont l’activité implique un engagement relationnel (TRUCHOT D. (2004). Epuisement professionnel et burnout : concepts, modèles, interventions, Paris, DUNOD).
La puissance d’évocation du terme burnout reflète la réalité des personnes qui expérimentent des souffrances en raison de leur travail.
Le burnout est un champ de recherche important et controversé (MASLACH C., SHAUFELI W.B. et LEITER M.P. (2001). Job burnout, Annu. Rev. Psychol., 52, 397-422).

Vocabulaire anglo-saxon, justification de cette utilisation
La traduction française = usure professionnelle, fatigue professionnelle, épuisement au travail, syndrome d’épuisement des soignants. Elle ne rend compte que partiellement du concept de burnout (MAURANGES A. et CANOUÏ P. (2001). Le syndrome d’épuisement professionnel des soignants : de l’analyse du burnout aux réponses, Paris, DUNOD).
L’idée de vécu chronique de stress dans le cadre de son travail peut être approchée par les termes : usure, épuisement ; car ils renvoient à une idée de perte d’énergie et de force. Pourtant, ils augmentent le risque de confusion avec d’autres états notamment physiques (LEBIGOT F. et LAFONT B. (1985). Psychologie de l’épuisement professionnel, Annales médico-psychologiques, 143 (8), 769-775).
L’épuisement, terme couramment utilisé peut se prêter à différentes situations de la vie quotidienne : après une journée au travail, on se sent épuisé, vidé, fatigué. Ce qui amène à interroger la spécificité et la valeur scientifique de l’épuisement professionnel (SCARFONE D. (1985). Le syndrome de l’épuisement professionnel (burnout) : y aurait-il de la fumée sans feu ?, Annales médico-psychologiques, 143 (8), 110-122).
La terminologie française utilisée pour le burnout élargie à l’extrême le champ de ce que peut ouvrir ce concept. Il devient alors une catégorie fourre tout ce qui donne lieu à différentes revendications (MAURANGES A. et CANOUÏ P. (2001). Le syndrome d’épuisement professionnel des soignants : de l’analyse du burnout aux réponses, Paris, DUNOD) mais surtout perd sa spécificité, il n’est plus distinguable d’autres concepts comme la fatigue, l’insatisfaction au travail et la charge de travail. Il peut même cacher certains troubles psychologiques comme des épisodes dépressifs, ou même des troubles de la personnalité qui seraient plus facilement diagnostiqués en épuisement professionnel car identifiés dans le cadre du travail. Enfin, l’expression épuisement professionnel crée une confusion avec l’une des dimensions du burnout qui est l’épuisement émotionnel.
Le verbe burn out : griller, brûler, s’user à cause d’une demande excessives d’énergie ou de ressources. Il désigne la réduction en cendre d’un objet entièrement consumé, qui disparaît. Par extension c’est la combustion totale des forces, des énergies et des ressources. Dans le domaine aérospatial, le burnout reflète l’épuisement de carburant d’une fusée avec comme conséquence de la surchauffe du moteur et le risque d’explosion de l’engin. Par cette violence, on peut reconnaître le Karoshi ou Kaloshi japonais signifiant la mort par fatigue au travail (HAUTEKEEETE M. (2001). Comment définir un ensemble de concepts complexes : stress, adaptation et anxiété, in P. GRAZIANI, M. HAUTEKEETE, S. RUSINEK et D. SERVANT (ed.), Stress, anxiété et troubles de l’adaptation, Paris, Acanthe/Masson, p.1-13).
 On préfèrera alors conserver l’utilisation de la terminologie anglo-saxonne afin d’éviter la confusion. 

Historique de l’évolution du concept de burnout
Les premières recherches portant sur le burnout étaient clairement exploratoires et avaient pour objectif de comprendre les personnes souffrant de leur travail (TRUCHOT D. (2004). Epuisement professionnel et burnout : concepts, modèles, interventions, Paris, DUNOD). Au début des années soixante-dix, de nombreux auteurs ont commencé à publier leurs observations. Elles ont porté au début sur la description du phénomène, lui donner un nom, et démontrer qu’il ne s’agit pas d’un phénomène transitoire. Ces auteurs étaient des chercheurs et ou des cliniciens impliqués dans un travail auprès de personnes qui nécessitent des soins psychiques ou somatiques. Ils travaillent en hôpital ou dans des services sociaux qui ont comme caractéristiques de générer de nombreux stresseurs émotionnels ou interpersonnels. L’environnement initial d’identification du burnout jouera un rôle important dans sa définition. C’est le cas de Freudenberger (1974) qui fut l’un des premiers à écrire un véritable article sur le burnout. Il est psychologue dans une clinique qui accueille des toxicomanes. Il remarque qu’un nombre important de soignants perdent leur dynamismes, leur engagement et leur motivation. Ce phénomène apparaît chez les individus qui étaient au départ très enthousiastes et qui au bout de quelques années se plaignent de douleurs physiques, de fatigue et d’épuisement (FRENDENBERGER H. (1977). Burn-out : the organizational menace, Training and developpement journal, 31 (7), 26-27). L’auteur relève une varité d’expressions et de manifestations de ce qu’il nomme alors burnout, métaphore de l’effet de la consommation de drogue. Il suggère que les pressions et les exigences professionnelles exercées sur les ressources d’un individu finissent par le conduire à un important état de frustration et de fatigue. Le professionnel s’épuise alors en essayant de répondre à certaines obligations imposées soit par son milieu de travail soit par lui-même. 
Maslach (1976), psychologue sociale, a découvert le burnout dans le cadre d’une recherche sur le stress émotionnel et les stratégies de coping développés par les employés des services sociaux face à leurs usagers. Les recherches sur le burnout prennent donc leur origine dans des services d’aide sociale ou de soins dont la caractéristique principale réside dans la relation  entre un « aidant » et un usager. La nature spécifique de ce type de profession place directement le burnout comme une transaction relationnelle entre une personne et une autre, plutôt que comme une réponse individuelle de stress. 
D’un côté les psychologues cliniciens se focaliseront sur la symptomatologie du burnout et les questions de santé mentale. De l’autres, les psychologues sociaux s’intéresseront au contexte professionnel. De ce fait, la majorité des premières études sont descriptives et qualitatives. Elles utilisent des techniques comme l’entretien, des études de cas ou l’observation participante (MASLACH C. (2001). What we learned about burnout and health ? Psychology and Health, 16, 607-611). Les études qui ont suivi étaient quantitatives et expérimentales pour comprendre les causes du burnout. 
Les résultats de ces études étaient : certains facteurs généraux qui suggèrent que le burnout présente des éléments communs et identifiables. En premier, le fait de procurer une aide sociale ou des soins s’avère très coûteux en énergie et génère un épuisement émotionnel qui devient une réponse commune devant cette charge de travail. En deuxième, une forme de désinvestissement, de dépersonnalisation ou de cynisme émerge des recherches menées avec des entretiens auprès de sujets qui tentent de gérer leur stress professionnel. Enfin, pour diminuer leur compassion envers les patients, certains sujets présentent un détachement émotionnel afin de se protéger des émotions intenses qui les empêchent d’accomplir leur travail. Ce détachement peut aboutir à des attitudes négatives voir déshumanisées.
Les caractéristiques de la relation « aidant/aidé » apportent un éclairage complémentaire au concept naissant de burnout. Le nombre d’usagers, l’importance des échanges ou le manque de ressources sont des facteurs impliqués dans le développement du burnout. Les relations avec les collègues, avec la famille des usagers sont également en lien avec le burnout. Devant ces découvertes qualitatives tout un corpus d’analyses et de recherches a été mené pour asseoir la valeur scientifique du concept. 
A partir des années quatre-vingt, les recherches sur le burnout prennent une dimension plus systématique, empirique et quantitative, en ayant recours à des questionnaires et des études méthodologiques sur un grand nombre de sujets (CORDES C. et DOUGHERTY T.W. (1993). A review and an integration of research on job burnout, Academy of Management Review, 18 (4), 621-656). 
Le questionnaire qui présente les qualités psychométriques les plus solides et qui est le plus utilise est l’inventaire du burnout de Maslach, mis au point par Maslach et Jackson (1986) (MASLACH C. et JACKSON S.E. (1986). The measurement of experienced burnout, Journal of occupational behaviour, 2, 99-113). Il a été validé en français par Dion et Tessier (1994) (DION G. et TESSIER R. (1994). Validation de la traduction de l’inventaire d’épuisement professionnel de Maslach et Jackson, Revue canadienne des sciences du comportement, 26, 210-227). 
Le burnout va permettre de connaître les contributions théoriques et méthodologiques d’autres champs de la psychologie, notamment de la psychologie du travail et des organisations. Dans la lignée du stress professionnel, les questions de satisfaction professionnelle, d’engagement professionnel ou de turnover sont appréhendées. La perspective clinico-sociale initiale du burnout va donc gagner de nouvelles perspectives et les études vont apporter une validité complémentaire au concept dans les années quatre-vingt-dix (GOLOMBIEWSKI R.T., BOURDEAU R.A. et LUO H. (1994). Global burnout : a worldwide pandemic explored by the model phase, Greenwich, JAI Press). 
En effet la phase empirique se poursuit dans de nouvelles directions. D’une part, le concept  n’est plus uniquement centré sur les professionnels des soins ou de l’aide sociale mais voit une extension dans d’autres champs professionnels (militaires, cadres d’entreprises, avocats …). D’autre part, les études deviennent plus solides et pointues avec l’utilisation d’outils statistiques et de modèles structuraux pour formaliser la relation complexe entre les facteurs organisationnels et les trois dimensions du burnout que sont l’épuisement émotionnel, le désinvestissement de la relation à l’autre et le sentiment d’inefficacité personnelle (LOUREL M., GANA K., PRUD’HOMME V. et CERCLE A. (2004). Le burnout chez les personnels des maisons d’arrêt : test du modèle « demande-contrôle » de Karasek, L’encéphale, XXX, 557-563). 
Cette approche permet aux chercheurs d’examiner simultanément l’influence et les conséquences d’un grand nombre de variables et de pouvoir identifier la contribution de chaque variable et de pouvoir identifier la contribution de chaque variable séparément sur le développement du burnout. Enfin depuis quelques années, les études longitudinales commencent à évaluer l’impact du milieu professionnel sur la santé mentale d’une personne (MC MANUS I.C., WINDER B.C. et GORDON D. (2002). Dissociation and posttraumatic stress disorder : two prospective studies of motor vehicle accident survivors. British journal of psychiatry, 180, 363-368). Ce qui permet de réflechir à des thérapies visant à soigner le burnout (GUERITAULT-CHALVIN V. et COOPER C. (2004). Mieux comprendre le burnout professionnel et les nouvelles stratégies de prévention : un compte rendu de la littérature, Journal de thérapie comportementale et cognitive, 14 (2), 59-70). (COTE L., EDWARDS H., BENOIT N. (2005). S’épuiser et en guérir : analyse de deux trajectoires selon le niveau d’emploi, Revue internationales sur le travail et la société, 3 (2), 835-865). 

Définition du concept de burnout
La définition est différente suivant l’approche : psychologie clinique, psychologie sociale, psychologie du travail, psychologie organisationnelle.
Schaufeli et Enzman ( SCHAUFELI W.B. et ENZMAN D. (1998). The burnout companion to study and practice, Palo Alto, Taylor and Francis) regroupent les définitions dans deux categories, celles qui envisagent le burnout comme un processus et celles qui le considérent comme un état.
Le burnout comme un processus :
Le burnout apparaît avec les tensions dues à un décalage entre les attentes ou les efforts du sujet et les exigences du travail (CHERNISS C. (1980). Staff burnout. Job stress in the human services, Beverly Hills, Sage) (EDLEWICH J. et BRODSKY A. (1980). Burnout : stages of disillusionment in the helping professions, New York, Human sciences press) 
Le stress professionnel se développe progressivement. Il peut amener à un état de mal-être. Les stratégies que met en place l’individu face à ce stress sont déterminantes dans le développement éventuel du burnout. Le mal-être s’exprime par un désengagement de l’employé face à son travail. Il ne peut plus faire face aux tensions ressenties (CHERNISS C. (1980). Staff burnout. Job stress in the human services, Beverly Hills, Sage).
Le burnout = processus signifie que c’est un stade final de mal-être qui se développe progressivement avec l’accumulation continue de stress chroniques face auquel le coping n’est pas efficace. 
Eldwich et Brodsky (1980), proposent un découpage en quatre stades : l’enthousiasme, la stagnation, la frustration et l’apathie. Au cours du premier (enthousiasme) : le sujet fait l’expérience d’un enthousiasme important, qui se traduit par une tendance à se rendre disponible de façon excessive et d’avoir des attentes irréalistes concernant son travail. Dans le second stade (stagnation), l’employé ressent une impression de stagnation durant laquelle ses attentes professionnelles deviennent plus réalistes. Un certain mécontentement personnel commence à faire surface comme le sentiment que le travail ne peut pas compenser ce qui manque dans sa vie. Au cours du troisième stade (frustration) les difficultés professionnelles semblent se multiplier et le sujet commence à remettre en question ses compétences. Il s’ennui, devient intolérant, moins à l’écoute des autres et tente de faire face à ces situations en les fuyant et en évitant ses collègues. Finalement, il en arrive au stade de l’apathie. Elle se caractérise par un état de dépression et d’indifférence en réponse aux frustrations répétitives auxquelles il se trouve confronté. Ce stade présente l’essence même du burnout pour Eldwich et Brodsky (1980).
Veniga et Spradley (1981) (VENIGA R.L. et SPRADLEY J.P. (1981). The work/stress connection : how to cope with job burnout, Boston, Little, Brown) proposent un processus semblable qui identifie cinq étapes dans le processus du burnout. Le premier est appelé honeymoon, et correspond à la première phase du processus précédemment décrit. Une baisse d’énergie et insatisfaction donnent naissance à  la deuxième phase et l’accentuation des stratégies d’évitement et de symptômes d’épuisement caractérise la troisième phase. La quatrième étape apparaît avec des symptômes critiques. C’est la crise durant laquelle le sujet devient pessimiste et tente de fuir son travail. L’étape finale,le mur, est atteinte lorsque le burnout est indissociable d’autres troubles (addiction aux drogues ; à l’alcool, troubles cardiaques…).
L’intérêt de cette conception c’est d’offrir une vision transactionnelle du burnout. Il est alors considéré comme le produit d’une relation où l’individu et l’environnement ne sont pas disjoints mais des composants qui s’influencent mutuellement et continuellement. Par ailleurs, elle donne une illustration visuelle très claire du burnout. Il est en effet aisé de se représenter une personne qui au départ très enthousiaste par son nouvel emploi, finit au bout de quelques années par présenter un état d’apathie. 
Pour autant, le burnout se présent-il systématiquement par stade ? existe-t-il des stades distincts et identifiables qui mènent au burnout ? peuvent-ils être mesurés ? les études montrent qu’il est extrêmement difficile de trancher ces questions et ne pas u répondre comporte un grand risque.  Le risque de faire du burnout un concept vague, difficilement identifiable et qui perdrait sa spécificité puisqu’on en viendrait à en faire une forme de stress professionnel. Il n’en demeure pas moins que l’idée de processus est intéressante pour le clinicien qui rencontre des personnes présentant un burnout (COTE L., EDWARDS H. et BENOIT N. (2005). S’épuiser et en guérir : analyse de deux trajectoires selon le niveau d’emploi, Revue internationale sur le travail et la société, 3(2), 835-865). En plus d’apporter une illustration plus simplifiée, la présentation du burnout comme l’enchaînement de différents stades donne aux patients des clés de repérage du développement de leur souffrance, il permet une identification des signes précurseurs du burnout. 

Le burnout comme état : un épisode psychopathologique ?
La majorité des chercheurs définissent le burnout comme un épisode ou un état même si ces conceptions varient en fonction de leur précision, de leurs dimensions ou de leur étendue (TRUCHOT D. (2004). Epuisement professionnel et burnout : concepts, modèles, interventions, Paris, DUNOD).
Trois composantes permettent de définir le burnout comme un état : la première concerne des éléments dysphoriques qui prédominent comme l’épuisement émotionnel, la fatigue, les cognitions dépressives (désespoir et impuissance) et les pensées négatives à l’égard de soi (MASLACH C. et SCHAUFELI W.B. (1993). Historical and conceptual developpement of burnout, in W.B. SCHAUFELI, C. MASLACH et T. MARCK (ed.), Professional burnout : recent developments in the theory and research, Washington, Taylor et Françis, p.1-16). Les personnes manifestent des comportements négatifs voire hostiles vis-à-vis d’autrui et sont moins efficaces ou productives. 
La seconde est d’ordre étiologique, ce qui signifie que l’on peut attribuer le burnout à des attentes inappropriées ou des exigences émotionnelles excessives ( SCHAUFELI W.B. et ENZMAN D. (1998). The burnout companion to study and practice, Palo Alto, Taylor and Françis). 
La troisième caractéristique concerne l’origine des symptômes : ils ne sont pas consécutifs à la présence d’une pathologie mentale chez le sujet, mais causés par son environnement de travail (les relations qu’il y tisse notamment) ou la perception qu’il en a. 
Le modèle prototypique de l’approche du burnout comme état le plus connu et le plus solide sur le plan expérimental est l’attributional environmental model (modèle attributionnel et en environnemental) de Maslach (1982) (MASLACH C. (1982). Understanding burnout : definitional issues in analysis a complex phenomenon, in W.S. PAINE (ed.), Job stress and burnout : research, theory and intervention perspectives, Beverly Hills, CA, Sage Publication, p. 29-40). Il s’est nettement démarqué des autres en devenant une référence pour la recherche sur le phénomène de burnout (MASLACH C, SCHAUFELI W.B. et LEITER M.P. (2001). Job burnout, Annu.Rev. Psychol., 52, 397-422). Dans ce modèle dont les points forts tiennent dans ses excellentes bases théoriques et empiriques, le burnout est défini comme un syndrome multidimensionnel comprenant trois composantes principales. 
La première caractéristique du burnout est l’état d’épuisement émotionnel. Il s’agit d’une absence quasi-totale d’énergie. Le sujet sent que ses réserves d’énergie sont complètement épuisées et qu’il n’est plus capable d’apporter son assistance à autrui sous quelque forme que ce soit. Ce manque d’énergie est d’autant plus fort que l’individu pense qu’il n’a aucun moyen à sa disposition pour recharger ses batteries. La seule pensée d’avoir à affronter une nouvelle journée au travail dans ces conditions lui est insupportable. Cette composante d’épuisement émotionnel représente la dimension stress du burnout. 
La deuxième caractéristique du burnout concerne un état de désinvestissement ou de désengagement de la relation à l’autre. Elle se traduit par une attitude négative et détachée de la part de la personne envers ses clients, patients ou collègues qui finissent par être traités comme des objets. Ce détachement excessif est souvent accompagné par une perte d’idéalisme. La composante de désinvestissement-dépersonnalisation correspond à la dimension interpersonnelle du phénomène de burnout. 
La troisième caractéristique du burnout tient en une diminution du sens de l’accomplissement et de la réalisation de soi ou en une forme de sentiment d’inefficacité personnelle. Le sujet va porter un regard particulièrement négatif et dévalorisant sur la plupart de ses accomplissements personnels. Cette perte de confiance en soi résultant de ce type de d’attitude est associée à des état dépressifs importants et à une incapacité à faire face aux obligations professionnelles. Cette forte sensation d’être inefficace peut aboutir à long terme sur un verdict d’écher que l’individu s’impose à lui-même et dont les conséquences peuvent être particulièrement graves tant pour l’employé que l’organisme professionnel dans lequel il travaille. La composante de diminution du sens d’accomplissement représente la dimension d’auto-évaluation du burnout. 

Du processus de stress à l’état de burnout
Le modèle initial de Maslach et Jackson (1981) a connu des remaniements importants. L’idée de signification existentielle en ce sens que le travail est une quête existentielle (PINES A., ARONSON E. et KAFRY D. (1981). Burnout : from tedium to personnal growth, New York, Free Press) est intégrée comme facteurs générant le burnout. C’est cependant de manière plus manifeste avec l’utilisation des modèles cognitivo-émotionnels, notamment le modèle transactionnel du stress de Lazarus et Folkman (1984) (LAZARUS R.S. et FOLKMAN S. (1984). Stress, appraisal and coping, New York, Springer) que ce modèle du burnout va acquérir une valeur exlicative. Ainsi, dans l’approche transactionnelle, le stress est considéré comme une transaction entre la personne et l’environnement que le sujet évalue comme débordant ses ressources et compromettant son bien-être. Le stress ne peut donc naître que parce que le sujet évalue subjectivement la situation comme menaçante ou parce que qu’il n’a pas de capacités (coping) pour la gérer. 
Pour faire le lien entre état et processus, Maslach et Schaufeli (1993) suggérent que le burnout se développe au fur et à mesure que les obligations professionnelles deviennent plus fortes et plus lourdes. Elles épuisent alors les ressources personnelles et l’énergie de l’individu. Le désinvestissement ou désengagement peut être considéré comme un coping permettant à la personne de prendre une distance psychologique vis-à-vis des usagers. Le but consiste à se protéger des effets négatifs et l’épuisement émotionnel  dont elle est victime. Pour finir, l’individu ressent une diminution de son sentiment d’accomplissement.  Il prend conscience du décalage existant entre son attitude et ses comportements actuels ainsi qu’entre les attentes qu’il pouvait avoir en débutant sa carrière et les contributions positives qu’il aurait pu faire aussi bien pour lui-même que pour son entreprise (CORDES C.L. et DOUGHERTY T.W. (1993). A review and an integration of research on job burnout, Academy of management review, 18 (4), 621-656). 
Enfin, contrairement aux conceptions initiales qui restreignaient le burnout aux professions de la santé et de l’aide sociale, le burnout est identifié dans toutes les professions dans les lesquelles les sujets sont engagés dans les relations avec autrui (LEITER M.P. et SCHAUFELI W.B. (1996). Consistency of the burnout construct across occupations, Anxiety, stress and coping, 9, 229-243). 
En conséquence, si la conception tridimentionnelle du burnout est conservée, c’est avec Maslach et Leiter (1997) (MASLACH C. et LEITER M.P. (1997). The truth about burnout : how organization cause personnal stress and what to do about it, San Francisco, CA, Jossey-Bass) que sa definition connaît des changements. Si l’épuisement émotionnel demeure fidèle à sa conception initiale, le désinvestissement est à présent à considérer comme une forme de désengagement de son travail générant des attitudes cyniques à l’égard de soi, d’autrui et de la sphère professionnelle. Enfin, la réduction de l’efficacité qui reflète la diminution du sentiment d’efficacité personnelle, le manque d’accomplissement et le manque de productivité. 
Il n’est donc pas étonnant avec cette nouvelle conceptualisation, qu’une extension des recherches sur des groupes non professionnes ait pu voir le jour. Le burnout a donc été étudié chez les sportifs (CRESSWELL S.L. et EKLUND R.C. (2004). The athlete burnout syndrome : possible early signs, Journal of sciences and medecine in sport, 7 (4), 481-487) , des soldats (OSCA A., GONZALEZ-CARMINO G., BARDERA P. et PEIRO J.M. (2003). Estès de rol y su influencia sobre bienestar fisico y psychoquico en soldas professionals, Psicothema, 15 (1), 12-17), ou encore des couples (WESTMAN M. et ELTZION D. (1995). Crossover of stress, strain and ressources from spouse to another, dans Journal of organizational behaviour, 16 (2), 169-181) et des femmes au foyer (KULIK L. et RAYYAN F. (2003). Spousal relation and well-being : a comparative analysis of Jewish and Arab dual-earner families in Israel, Journal of community psychology, 31 (1), 57-73). 
Une question se pose donc: quelles sont les causes du burnout?

Les causes du burnout
Il a été initialement identifié dans le cadre professionnel, il est donc évident que le burnout soient causés par des facteurs relatifs au travail et à l’organisation de l’entreprise. Ces facteurs vont jouer un rôle important dans la souffrance des employés. Toutefois, deux personnes travaillant dans la même entreprise ayant les mêmes compétences et les mêmes tâches ne présenteront pas forcément un burnout. Ce sont donc les variables inter et intra-individuelles qui donneront une indication de l’expression possible de burnout chez un sujet. Bien entendu, il semble que la relation à l’autres lorsqu’elle est évaluée comme chronique et stressante est une base étiologique du burnout, que cette relation ait lieu dans un espace professionnel ou personnel. Malheureusement, peu d’études ont été menées sur les caractéristiques du burnout dans un milieu non professionnel.
Les variables professionnelles et organisationnelles : 
En majorité les recherches sur la contribution de variables relatives au travail et à l’organisation sur le burnout se situent au niveau du rapport direct entre le sujet et son environnement. La structure hiérarchique, le style de management ou la structure de l’institution sont rarement pris en compte, les entreprises étant particulièrement réticentes à laisser une liberté au chercheur d’interroger le lien entre mode de management et santé de employés (TRUCHOT D. (2004). Epuisement professionnel et burnout : concepts, modèles, interventions, Paris, DUNOD).
En raison d’une plus grande facilité d’opérationnalisation, ce sont les variables directes de l’activité, du contenu de la tâche et de son contexte qui ont été étudiés. Le burnout, notamment sa dimension épuisement émotionnel est corrélé avec la charge (lourdeur des horaires) et le rythme de travail (imprévisibilité et fréquence) (GREENGLASS E.R., RONALD J.B. et MOORE K.A. (2003). Reaction to increased workload : effects on professionnal efficacy of nurses, Applied Psychology : an international review, 52 (4), 580-597) (MASLACH C, SCHAUFELI W.B. et LEITER M.P. (2001). Job burnout, Annu.Rev. Psychol., 52, 397-422). Cependant des analyses plus fines montrent que ce qui sous-tend l’expression du burnout ce n’est pas tant la réalité objective des demandes du travail que le sentiment de les contrôler (TRUCHOT D. et BADRE D. (2004). Agressions au travail et burnout chez les travailleurs sociaux : influence de l’auto-efficacité et du pouvoir organisationnel, dans stress et trauma, 4 (3), 187-194). En effet le sentiment de contrôle et d’autodetermination sur son travail se révèle un meilleur prédicteur du burnout (FERNET C, GUAY F. et SENECAL C. (2004). Adjusting to job demands : the role of work self-determination and job control in predicting burnout, dans journal of vocational behaviour, 65 (1), 39-56). 
Un autre corps d’étude sur les variables du travail a trait aux caractéristiques du contexte de travail. En premier, les conflits de rôle c'est-à-dire lorsque les informations requises pour effectuer une mission sont contradictoire ou encore l’ambiguïté de rôle lorsque ces informations sont inadéquates ou insuffisantes, sont des facteurs médiateurs du burnout. Mais c’est avec l’étude sur le support social que les relations ont été mieux spécifiées. En effet, un manque de soutien des supérieurs hiérarchiques notamment, mais aussi de la part des collègues augmente la vulnérabilité devant le burnout. Il s’agirait d’un effet tampon qui modulerait la relation entre les stresseurs professionnels et le burnout (SCHAT A.C.H. et KELLOWAY K.E. (2003). Reducing the adverse consequences of workplace agression and violence : the buffering effects of organizational support, journal of occupational health psychology, 8 (2), 110-122). Le contexte de travail, les demandes professionnelles, le support social et le sentiment de contrôle sont donc des facteurs importants dans le développement du burnout. Ils montrent que le sujet et la relation à l’autre sont au cœur du développement du burnout, d’où l’importance de l’étude des variables inter et intra-individuelles.

Les variables inter et intra-individuelle
Les personnes n’évoluent pas de manière robotisée sur leur lieu de travail ; elles construisent psychiquement les situations et les relations avec autrui. Elles apportent donc leurs capacités, leurs qualités, leurs difficultés et leur vision du monde. Des facteurs interindividuels sont donc présents ainsi que des facteurs personnels. Ces derniers sont étudiés sous deux angles, les facteurs démographiques et ceux liés à la personnalité. Les facteurs interindividuels qui génèrent un burnout s’articulent autour des victimisations vécues au travail. Les situations d’agressions, de conflits avec les usagers, les brimades de la part des collègues, le harcèlement…, contribuent à l’expression du burnout (KOP N., EUWEMA M. et SCHAULEFI W.B. (2003). Burnout, job stress and violent behaviour among Dutch police officiers, Work and stress, 13 (4), 326-340) (TRUCHOT D. et BADRE D. (2004). Agressions au travail et burnout chez les travailleurs sociaux : influence de l’auto-efficacité et du pouvoir organisationnel, dans stress et trauma, 4 (3), 187-194) (VARTIA M. et HYYTI J. (2002). Gender diffrences in workplace bullying among prison officiers, dans European journal of work and organzational psychology, 11 (1), 113-126). 
L’âge, le sexe, le niveau d’éducation ou le statut marital montrent des relations inconstantes avec le burnout. Ainsi, les recherches se sont orientées sur les facteurs de personnalité. On retrouve des résultats comparables à ceux effectués pour le stress. En effet un niveau de hardiesse important, un lieu de contrôle interne et des coping centrés sur le problème sont associés à des niveaux faibles de burnout. 
Il faut cependant souligner que la majorité des études portent presque exclusivement sur des employés qui ne présentent pas de burnout (WEBER A. et JAEKEL-REINHARD A. (2000). Burnout syndrome : a disease of modern societies ?, dans Occupational medecine, 50 (7), 512-517). La grande majorité des études se focalisent sur certaines variables organisationnelles ou individuelles auprès de personnels qui sont en poste et donc en état de travailler. 
Il est donc plausible que certaines personnes souffrant d’un burnout pathologique ne participent pas à ces études. On peut l’imaginer dans la mesure où l’épuisement émotionnel et le désengagement provoquent des attitudes cyniques, une fatigue intense. Elles seraient donc démotivées, ne repéreraient aucun accomplissement dans leur travail et se sentiraient inefficaces. Elles sont parfois dans une souffrance telle qu’elles peuvent ne plus être en poste (BAKKER A.B., DEMEROUTI E., DE BOER E. et SCHAUFELI W.B. (2003). Job demands and job ressources as perdictors of abscence duration and frequency, Journal of vocational behaviour, 62 (2), 341-356) ou sont en arrêt pour cause de maladie. On peut donc entrevoir une explication des résultats inconsistants des variables individuelles et surtout s’interroger sur l’évaluation du burnout. 

Le burnout : un trouble psychopathologique ?
Comment évaluer et distinguer le burnout du trouble de burnout ?
S’il existe un consensus sur la composition du burnout, c'est-à-dire la présence de trois dimensions que sont l’épuisement émotionnel, le désinvestissement et la dimension de l’efficacité, le problème de l’évaluation clinique reste posé. 
Il existe bien des outils pour mesurer le burnout (SCHAUFELI W.B., BAKKER A.B., HOOGDUIN K., SCHAAP C. et KLADLER A. (2001). On the clinical validity of the Maslach burnout inventoryand the burnout measure, Psychology and health, 16, 565-582), largement utilise dans la communauté scientifique. Il s’inscrit dans une approche dimensionnelle et chaque sujet obtient une note sur chaque dimension du burnout. Le MBI permet d’établir un profil de burnout, c'est-à-dire que le sujet obtient trois notes que l’on compare à des normes (LIDVAN-GIRAULT N., (1989). Burnout : émergence et stratégie d’adaptation. Le cas de la médecine d’urgence, Paris, Université Paris 5 René Descartes). Cependant, aucun seuil n’est établi pour diagnostiquer un burnout pathologique qui caractériserait l’état d’une personne présentant une souffrance psychique intense (BOUDOUKHA A.H. (2006). Etude conjointe du burnout et des troubles de stress post-traumatique dans une population à risques, dans Cas des professionnels en milieu carcéral, Lille 3 Charles de Gaulle, Villeneuve d’Ascq). 
La définition de Maslach et Leiter (1997) a permis un développement de recherches organisationnelles. En contrepartie, les recherches cliniques ont été délaissées. En effet, les manifestations psychologiques du burnout n’ont pas connu d’évaluations systématiques (COTE L, EDWARDS H. et BENOIT N. (2005). S’épuiser et en guérir : analyse de deux trajectoires selon le niveau d’emploi, revue internationale sur le travail et la société, 3 (2), 835-865). Le repérage du trouble de burnout ou de burnout pathologique pose différents problèmes sur le plan de sa symptomatologie. D’une part, les classifications internationales qui font consensus (DSM ou CIM) ne proposent pas de diagnostic de burnout pathologique ou de trouble de burnout, contrairement aux traumatismes psychologiques. Schaufeli et Enzman (1998) repèrent plus de 130 symptômes dans les articles parus sur la question. Ce nombre élevé est lié à la nature des premières recherches sur le burnout essentiellement descriptives. 

Evaluation des symptômes du burnout
Cette abondance de symptômes amène Cordes et Dougherty dès 1993 à proposer un regroupement des symptômes en cinq catégories : physiques, émotionnels, interpersonnels, attitudinaux et comportementaux. Ils peuvent être intégrés aux indicateurs (objectifs et subjectifs) qui, pour Bibeau et al. (1989) (BIBEAU G., DUSSAULT G., LAROUCHE L. M., LIPPEL K., SAUCIER J.F., VEZINA M. et al. (1989). Certains aspects culturels diagnostiques et juridiques du burnout, pistes et repères opérationnels, Montréal, GIRAME, Université de Montréal) permettent d’établir un diagnostic différentiel.
Des indicateurs objectifs : reflètent les symptômes interpersonnels et comportementaux de Cordes et Dougherty (1993). Il s’agit d’une diminution significative du rendement et des négligences au travail (Shanafelt, Bradley, Wipf et Balck, 2002). Ils s’accompagnent d’une insatisfaction (Woplin, Burke, Greenglass, 1991) (WOPLIN J., BURKE R.J. et GREENGLASS E.R. (1991). Is job satisfaction as antecedent or a consequence of psychological burnout ? dans Human Relation, 44, 193-209). On observe également une mise à distance, un cynisme ou un désengagement des relations avec les usagers (Truchot et Badré, 2003) (TRUCHOT D. et BADRE D. (2003). Equity and the burnout process : the role of helping models, dans Revue internationale de psychologie sociale, 16 (4), 5-24). 
Des indicateurs subjectifs, plus nombreux et reflètent des symptômes physiques, psychiques et émotionnels. On retrouve un état de fatigue marqué (Shirom, Melamed, Toker, Berliner et Shapira, 2005) (SHIROM A., MELAMED S., TOKER S., BERLINER S. et SHAPIRA I. (2005). Burnout, mental and physical health : a review of the evidence and proposed explanory model, dans International Review of Industrial and Organaizational Psychology, 20, 269-309) associé à une perte de l’estime de soi (Kahill, 1988) et des symptômes dysphoriques comme le désespoir ou l’anxiété (Brenninkmeijer, Van Ypere, et Buunk, 2001). Des somatisations multiples apparaissent sous forme de plaints physiques ou mentales (De Vente, Olff, Vam Amesterdam, Kamphuis et Emmelkamp, 2003). Enfin, se développement des difficultés de concentration, une irritabilité quotidienne et un négativisme (Kahill, 1988).

Proposition d’évaluation et de diagnostic pour le trouble de burnout ou burnout pathologique
La classification précedement présentée suscite deux problèmes majeurs : elle inscrit de manière trop rigide le burnout comme une souffrance professionnelle occultant les groupes  non professionnels chez lesquels le burnout est observé ; elle ne distingue pas le burnout du burnout pathologique. Il semble alors primordial de pouvoir faire une distinction des deux formes de burnout : 
Le burnout en tant qu’objet de recherche : il s’agit de la souffrance plus ou moins intense qu’expriment les professionnels, notamment en raison des situations chroniques de stress auxquels ils sont confrontés. On trouverait donc des personnes plus ou moins burnoutées. Ce burnout a un ancarge dans le champ de travail.
Le burnout en tant que concept psychopathologique. Il désigne les personnes qui souffrent tellement des stress chroniques auxquels elles sont exposées que leur état de mal-être clinique nécessite une prise en charge psychothérapeutique. Pour le distinguer du précedent on utilise le terme de burnout pathologique, burnout dysfonctionnel, de burnout clinique ou trouble de burnout. 
Les travaux cliniques de Boudoukha (2006) ont permit de retenir ces deux formes de burnout. 
Le burnout pathologique se distingue d’une souffrance professionnelle. Il s’agit d’une souffrance psychique consécutive de stress relationnels pathogènes pour lesquels le sujet n’a aucun coping disponible, lui donnant l’impression d’être emprisonné, piégé, enfermé dans une relation. On peut le considérer comme l’état final d’un processus d’accumulation chronique de stress relationnels. Cet état final à travers trois dimensions : 
Une dimension émotionnelle : le patient n’est plus en mesure de moduler sa gamme d’émotions habituelles. Il ressent une impression d’incapacité à exprimer des émotions notamment positives. La fatigue émotionnelle est patente, donnant au tableau clinique une impression de froideur ou de tristesse. Le patient exprime que ressentir ou exprimer des émotions lui coûte beaucoup d’énergie. 
Une dimension relationnelle : le patient se plaint de difficultés relationnelles. L’autre, auparavant source de joie ou de réconfort est devenu une source de stress, de problèmes ou de malaise. Le patient va fuir les relations, les interactions ou l’implication avec l’autre. Différentes attitudes vont apparaître pour éviter ou faire cesser les relations-interactions avec autrui. Cela peut être des attitudes méprisantes, cyniques, hautaines voire une réification.
Une dimension cognitive. On observe des plaintes relatives à un sentiment de fatigue intellectuelle. Par ailleurs, le contenu des pensées est focalisé autour du sentiment d’inefficacité personnelle. 

\section{article : job demands and job ressources as predictor of absence duration and frequency (Bakker A.B., Demerouti E., De Boer E. et Schaufeli W.B., 2003. Journal of vocational behaviour, 62, (2), 341--356)}
Absentéisme 
En général il existe deux mesures différentes : la fréquence de l’absence et sa durée (HENSING, ALEXANDERSON, ALLEBACK et BJURULF, 1998). La fréquence de l’absence est le nombre de période ou de fois où la personne a été absente durant une période déterminée indépendamment de la durée des périodes. En général il est utilisé pour indique l’absentéisme volontaire. Il est fonction de la motivation de l’employé. En opposition, la durée de l’absence est la période de temps totale où l’individu était absence indépendamment du nombre de périodes. Cet indicateur reflète l’absentéisme non volontaire qui résulte d’une incapacité plus que d’une réticence à aller au travail. La correlation entre la fréquence d’absence et la durée varie entre 0.05 (faible) et 0.60 (relativement élevée). 
La plupart des études empiriques centrées sur l’expérience professionnelle comme précurseur de l’absentéisme peuvent être classées selon deux explications de la décision des employés de se faire porter malade (JOHNS, 1997). En premier, les employés peuvent être absents car ils veulent éviter une situation professionnelle aversive (répulsive, déplaisante). Dans cette hypothèse il a été démontrée que les employés qui ont une satisfaction au travail et un engagement organisationnel faibles sont les plus fréquemment absents que ceux qui ont une satisfaction au travail et un engagement organisationnel élevés (COHEN,1991 ; FARELL et STAMM, 1998 ; MATHIEU et KOHLER, 1990 ; SAGIE, 1998). Dans ces études l’absentéisme est interprété comme une fuite de, une compensation pour ou bien une protestation contre une situation professionnelle aversive ou démoralisante. Ceci est en accord avec la notion d’absentéisme volontaire. 
La seconde explication de l’absentéisme est que ce comportement est une réaction au stress au travail. Lorsque le stress est conçu comme un échec pour répondre aux besoins du travail. cette explication stipules que l’absentéisme peut être utilisée comme un mécanisme de « coping » pour faire face aux tensions et ce n’est pas uniquement une réaction comportementale d’insatisfaction. De nombreux stresseurs (des facteurs liées au travail qui peuvent causer une réaction psychologique négative comme l’anxiété et la fatigue) comme par exemple la charge de travai, la monotonie (MELAMED, BEN-AVI, LUZ et GREEN, 1995), les problèmes de rôles (JAMAL, 1984) ont été associés avec des taux d’absence important. Par contre, Johns (1997) a observé que les études qui ont porté sur la relation entre les stresseurs et l’absentéisme sont nombreuses, mais celles qui intègrent les tests de modèles de médiation (stresseur              stress, réaction           absence) 

Engagement organisationnel et absentéisme
MEYER et ALLEN (1991) considèrent l’engagement comme un concept multidimentionnel incluant trios composantes : affective, normative et durée d’engagement. L’engagement affectif réfère aux employés émotionnellement attachés à, indentification avec et l’engagement dans une organisation. Par contre l’engagement normatif réfère à l’attachement des employés à l’organisation et ces objectifs à cause d’une idéologie ou d’un sentiment d’obligation. Finalement, l’engagement continuel ou sur la durée réfère à une crainte générale du coût lié au détachement d’une organisation ou bien de la perception des alternatives ou du degré de sacrifice par les employés. 
La plupart des études sur l’absentéisme ont examinée la corrélation des composantes affective et durée (GELLATLY, 1995)
L’engagement affectif qui devrait augmenter lors des experiences de travail personnellement gratifiantes, a toujours été lié de manière négative à l’absentéisme. Par contre, l’engagement de continuité paraît encourager le comportement d’absence. Comme l’a noté Brehm (1966), «  se sentir enfermé dans » peut provoquer une réactance (une réaction émotionnelle directement en contradiction avec les règles ou les régulations qui élimine un comportement spécifique de libérté) exprimée par de petits épisodes d’échappement. Cette relation positive entre l’engagement continuel et l’abenstéisme (composante fréquence) a été confirmé dans certaines études (Gellatly, 1995). Mais certaines études n’ont pas trouvé de relations. L’engagement normatif paraît stimuler la fréquentation à cause du sentiment d’obligation. Pourtant, il n’y a pas de preuves empiriques de la relation entre l’engagement normatif et l’absentéisme (Gellatly, 1995).
Suivant le paradigme du retrait, les personnes vont être plus en clin à se retirer des organisations dans les quelles il y a moins d’engagement.  De même, Farell et Stamm (1988) ont trouvé une corrélation moyenne corrigée de 0.12 entre la durée de l’engagement et la durée de l’absence dans leur méta-analyse incluant 11 échantillons. Ils ont trouvé une plus grande corrélation moyenne corrigée 0.23 lorsqu’ils se sont limités à 6 échantillons et qu’ils ont mesurés la fréquence de l’absence, ce qui s’accorde avec le raisonnement précédent : la fréquence de l’absence reflète surtout l’absence volontaire. En outre, Cohen (1991) rapporte une corrélation moyenne corrigée de 0.11 entre l’engagement et l’absence, résultats sur la base de 11 études
Ainsi, en général, la relation entre l’engagement organisationnel et l’absentéisme est plus faible avec la fréquence de l’absence qu’avec la durée de l’absence. 

Burnout et absentéisme
Le burnout peut être défini de manière général comme un syndrome de fatigue, de cynisme et réduction de l’efficacité professionnelle (Maslach, Jackson et Leiter, 1996). Alors que l’épuisement émotionnel et le cynisme (ou la dépersonnalisation) étaient considérés comme les dimensions principales du burnout, le sentiment de baisse de l’efficacité semble jouer un rôle différent. Pour le moment, la baisse de l’efficacité peut aussi être interprétée comme une possible conséquence du burnout (KOESKE et KOESKE, 1989 ; Shirom, 1989). En outer, il y a une accumulation d’évidence que l’accomplissement personnel est largement développé en parallèle avec les deux autres dimensions du burnout (Lee et Ashforth, 1996 ; Shaufeli et Enzmann, 1998). Ces résultats appuient la notion que l’épuisement émotionnel et la dépersonnalisation (ou cynisme) constituent un syndrome, qui est faiblement relié à l’efficacité professionnelle. C’est pour cette raison que cet aspect est exclue du modèle de recherche utilisé dans cette étude.
L’absentéisme est généralement considéré comme une conséquence importante du burnout au niveau organisationnel. Pourtant, l’épuisement émotionnel, la dépersonnalisation et la réduction de l’efficacité expliquent en moyenne pas plus que 2 pourcent de la variance de l’absentéisme (Lawson et O’Brien, 1994 ; Price et Spence, 1994). La revue de littérature de Schaufeli et Enzmann (1998) conclude pourtant à ce que : “…malgré l’hypothèse populaire selon laquelle le burnout cause l’absentéisme, l’effet est assez faible et plus lié à l’épuisement émotionnel » (p.91). Plusieurs méta-analyse de l’absentéisme montrent que le stress lié au travail n’est qu’une des nombreuses variables identifiées pour le comportement d’absentéisme chez les employés. Ainsi, on ne s’attend pas à ce que le stress au travail et l’absentéisme soient fortement corrélés (Beehr, 1995 ; Nicholson, 1993). Les variables non liées au travail conté pour l’absentéisme inclue un grand nombre de facteurs comme les caractéristiques personnelles, les blessures sportives, la consommation d’alcool, les désordres psychologiques et la douleur physique (Johns, 1997 ; Youngblood, 1984).  Ces variables non liées au travail peuvent aussi interagir avec les variables liées au travail et montrer la complexité de la relation avec l’absentéisme. Par exemple, dans leur étude sur 211 employés, parents mariés, Erickson, Nichols et Ritter (2000) ont trouvé que les obligations familiales modèrent l’effet du burnout de travail sur la fréquence de l’absence. Connaître de hauts niveaux de burnout a été associé avec l’augmentation de l’absentéisme si les employés avaient des enfants âgés de moins de 6 ans vivant avec eux ou bien ayant des difficultés avec les arrangements de la garde de leurs enfants. 

Le modèle : exigences de l’emploi-ressources
Au cœur de ce modèle (JD-R) (Demerouti et al., 2001) est l’hypothèse qu’alors que les employés dans différentes organisation peuvent être confronté à différents environnements de travail, les caractéristiques de ces environnements peuvent toujours être classés dans deux catégories générales : exigences du travail, ressources du travail, et ainsi constituer un modèle général qui peut être appliqué dans diverses situations indépendamment des exigences ou des ressources particulières. 
Les exigences du travail réfèrent aux aspects du travail, qu’ils soient physiques, psychologiques, sociaux ou organisationnels, qui nécessitent un effort soutenu (physique ou psychologique donc cognitif ou émotionnel), et par conséquent associés avec un certain coût physiologique et ou psychologique. Les exemples de cette situation : le travail sous haute pression, surcharge de rôle, conditions environnementales pauvres et des problèmes liés à la réorganisation.
Les ressources du travail sont les aspects physiques, psychologiques, sociaux ou organisationnels du travail qui sont soit/ou : fonctionnel dans l’accomplissement des objectifs de travail, réduire les exigences du travail et les coûts physiologiques et psychologiques associés, stimule le développement personnel. 
Les ressources peuvent être localisées au niveau de l’organisation de manière très large (salaire, opportunités de carrière, sécurité de l’emploi), au niveau interpersonnel (le soutien du chef et des collègues, ambiance de l’équipe), au niveau organisationnel du travail (clarté du rôle, participation dans la prise de décision), au niveau des tâches (signification des tâches, identité de la tâche, autonomie, variété des compétences)
Une seconde proposition dans le modèle JD-R est que les caractéristiques du travail peuvent évoquer deux différents processus. En premier, de hautes exigences de travail (exemple surcharge de travail) peut épuiser les ressources des employés sur un plan mental et physique et entraîner des problèmes de santé ou un burnout (Demerouti, Bakker, Nachreiner et Shaufeli, 2000, 2001 ; Lee et Ashforth, 1996, Leiter, 1993). En second, la pauvreté ou le manque de ressources du travail qui empêche d’atteindre l’objectif actuel ce qui cause une frustration et amène à un constat d’échec. Ceci pourrait induire un retrait du travail et réduire la motivation ou l’engagement. Lorsque les ressources de l’environnement extérieur viennent à manquer, les individus ne peuvent pas réduire l’influence potentiellement négative d’une exigence importante de leur travail et ne peuvent pas accomplir les objectifs de leur travail. Dans ce type de situation, réduire l’engagement peut être un mécanisme d’auto-protection  important  pour prévenir une future frustration de la non obtention des objectifs de travail (Antonovsky, 1987 ; Hackman et Oldham, 1976, 1980).
Les résultats de cette étude montrent qu’il faut différencier la durée et la fréquence de l’absence pour une meilleure gestion des ressources humaines. Ils suggèrent aussi que pour les réduire des mesures spécifiques doivent être prises en relation avec l’environnement de travail. De manière plus précise, cela réduira ou préviendra le burnout et ainsi réduire la durée des absences, les exigences du travail devraient alors être réduites ou mieux optimisées. De plus, afin d’augmenter l’engagement et réduire la fréquence de l’absence, la validité des ressources du travail devrait être considérée. Schaufeli et Enzman (1998) ont décrit d’importantes interventions au niveau organisationnel qui peuvent être utilisées pour atteindre ce but, incluant le redesign du travail, le coaching du travail et les programmes de développement organsiationnels. Pourtant, il serait plus facile d’influencer la fréquence de l’absence avec des outils de gestions que la durée de l’absence, depuis que réduire la charge de travail ou éliminer les process de réorganisation peuvent être difficile à réaliser dans certains cas, alors que la disposition de contrôle de travail et la montée de la participation des employés peut être plus facile à atteindre par des approches de redesign de travail. 

\section{articlle : s'épuiser et en guérir : analyse de deux trajectoires selon le niveau d'emploi, Côté L., Edwards H. et Benoit N. (2005). revue internationale sur le travail et la société, 3 (2), 835--865}
Les connaissances scientifiques actuelles sur le burnout, aussi appelé "épuisement professionnel", portent presqu'exclusivement sur les employés qui n'ont pas atteint un niveau critique dysfonctionnel (Weber et Jaekel-Reinhard, 2000). En effet, la plupart des études musrent certaines caractéristiques organisationnelles et personnelles auprès d'employés fonctionnels au travail. Le processus de guérison, en particulier, n'a fait l'objet que de rares recherches exploratoires, effectuées auprès de personnes travaillant, en très grande majorité, dans les professions où la relation d'aide occupe une place importante (Maslach et Shaufeli, 1993). Aucune étude comparative selon le niveau d'emploi n'a été recensée, dans quelle mesure les connaissances actuelles s'appliquent aux personnes occupant des emplois non professionnels et qui deviennent incapables de travailler en raison d'un burnout.
Définition du burnout
Malgré les nombreuses recherches effectuées sur le burnout, il n'existe pas de consensus sur la définition conceptuelle ou opérationnelle du burnout (Schaufeli, Maslach et Marek, 1993)  en fait la diversité des point de vue sème la confusion (Weber et Jaekel-Reinhard, 2000). 
La population générale, les chercheurs et les professionnels de la relation d'aide ne définissent pas le burnout de la même façon et leurs perspectives ne convergent pas toujours. En effet, les thérapeutes et psychologues cliniques travaillent principalement aurpès de personnes éprouvant des diffiecultés sur le plan professionnel ou exprimant une détresse psychologique, tandis que les chercheurs et les interenants aurpès des organisations tels que les psychologues industriels et organisationnels ont presque exclusivement accès à une main d'oeuvre professionnelle active. 
Les premiers voient donc danvantage des individus qui ont atteint un niveau critique ou clinique. Leurs écrits sont principalement basés sur leurx expériences professionnelles auprès des clients (exemple Lafleur, 1999); ils mettend davantage l'accent sur les facteurs presonnels et n'ont généralement pas une connaissance directe du milieu organisationnel. 
Les seconds rejoignent l'ensemble des employés présents dans leur lieu de travail; typiquement, ils distribuent des questionnaires visant à mesurer le niveau deburnout en relation avec des variables organisationnelles; ils ne connaissent généralement pas l'histoire personnelle des participants (par exemple, Golembiewski et al., 1996; Lee et Ashforth, 1996; Maslach, 1993; Maslach et Leiter, 1997) . 
La population générale pour sa part utilise le mot "burnout" pour désigner un malaise et un dysfonctionnement relié au travail. De plus, les différents points de vue sur le burnout sont parcellaires; il n'y a ni continuité ni unité entre eux. La présente étude vise à faire le pont entre les différentes perspectives et tente de combler un vide. 
Actuellement, le plus connu et le plus utilisé des modèles de burnout est celui de Maslach et Jackson (1981). Ce modèle comporte trois dimensions spécifiques au burnout : l'épuisement émotionnel, la dépersonnalisation et une diminution du sentiment d'accomplissement. En 1996, Maslach, Jackson et Leiter proposaient une nouvelle définition du burnout qui petu être employé pour tous les types de profession et qui comprend trois dimensions découlant du modèle original, soit l'épuisement émotionnel, le cynisme et l'éfficacité professionnelle. Le modèle de Maslach a été extrêmement utilse pour l'avancement de la recherche sur le burnout dans les organsiation, mais sa prépondérance a également eu pour effet de réduire la recherche sur les phases avancés du burnout, lesquelles entrent dans le domaine "clinique". 
Eldwich et Brodsky (1980) ont élaboré un modèle développemental du burnout a partir d'entrevues réalisées auprès d'employés désillusionnés mais fonctionnals dans des professions de relation d'aide. Leur modèle comporte quatre phases. La phase initiale, celle de l'enthousiasme, est liée aux grand espoirs à un niveau d'énergie élevé et à des attentes élevées. Durant la deuxième phase, la stagnation s'installe et le travail n'est plus suffisamment stimulant pour constituer le centre d'intérêt de la vie. L'employé tente alors de satisfaire ses besoins personnels : l'argent, la gestion du temps et le développement de carrière deviennent importants. Vient ensuite la frustration: non seulement les contraintes du travail nuisent--elles à la satisfaction des besoins personnels, mais elles menacent même la signification de ce que la personne accomplit. Des problèmes comportementaux, émotionnels et physiques peuvent alors surgir. La quatrième phase est caractérisée par un retrait dans l'apathie, un mécanisme typique et normal de défense face à la frustration. La personne investit le minimum de temps et d'effrot pour conserver un emploi qui lui apporte certes la sécurité, mais plus aucune satisfaction.
Veninga et Spradley (1981) suggèrent un processus similaire comprenant cinq phases ou étpaes. La première étape, celle de la lune de miel, correspond à celle de l'nethousiasme face au travail. L'étape deux commence lorsqu'il y a baisse d'énergie, insatisfaction et baisse de productivité au travail. Des stratégies d'évitement se manifestent graduellement. La troisème étape voit l'accentuation des symptomes de l'épuisement chronique, de la maladie physique, de la colère et de la dépression. L'étape quatre, celle de la crise, survient lorsque les symptômes deviennent critiques: l'employé est pessimiste et devient obsédé par ses frustrations; il développe une mentalité de fuite. Eventuellemnt, dans l'étape finale, la personne frappe le mur: le burnout devient alors indissiciable de divers autres problèmes tels que l'alcoolisme, la consommation de drogues, la maladie mentale ou les troubles cardiaques. La personne est déosrmais complètement vidée de toutes énergies et elle perd le controle de sa vie. Les dernières étapes de ce modèle correspondent à la conception du burnout retenue pour cette étude, car les employés touchés sont clairement dysfonctionnels. 
L'identifiaction du burnout
L'identifiaction des perosnnes vivant une atteinte à leur santé psychologique en raison de difficultés professionnelles qui les forcent à s'absenter du travail pour une période prolongée cause problème, vu que le terme "burnout" ne fait pas partie des diagnostics officiels du DSM-IV publié par l'American Psychiatric Association (1994), qui est la référence médicale pour les troubles mentaux. Les médecins ne posent don pas ce diagnostic. Typiquement, ils utilisent les diagnostics de dépression, d'anxiété ou de troubles de l'adaptation. Cette notion de burnout clinique, dans ce sens que la personne est devenue dysfonctionnelle ou qu'elle traverse la dernière phase du modèle développement de Veninga et Spradley (1981) ne fait pas l'unanimité; elle constitue un concept vague qui ne peut présentement être évalué de façon formelle par aucun instrument, par aucun procédure officielle. 
Pour mesurer les dimensions de leur modèle du burnout, Maslach et Jackson (1981) ont développé un instrument appelé le Maslach Burnout Inventory qui est largement utilisé dans la communauté scientifique. Le MBI s'inscrit dans une approche dimensionnelle et chacun se situe sur un continuum du burnout. Il permet d'établir un niveau de burnout pour les individus et le sorganisations, mais aucun seuil, aucun point de coupure n'a été établi pour avaluer le burnout type clinique (Truchot, 2004). 
Une autre approche pour indentifier le burnout consiste à analyser les manifestations symptomatiques. Maslach et Schaufeli (1993) ont effectué une synthèse des définitions pour établir cinq éléments diagnostiques applicables à la phase terminale du burnout : 1) prédominance des symptômes dysphoriques tels que l'épuisement émotionnel, la fatigue et la dépression, 2) l'accent est mis sur les symptômes mentaux et comportementaux plutôt que sur les symptômes physique, 3) les symptomes son reliés au travail, 4) les symptômes se manifestent chez une personne "normale" ne souffant pas à l'origine de psychopathologie; 5) une bassie d'efficacité et de productivité en raison d'une attitude négative et de comportements contreproductifs. 
Dans son étude sur les aspects cuturels, diagnostiques et juridiques du burnout, un comité médical dirigé par Bibeau et al. (1989) a établi un diagnostic différentiel se basant sur les indicateurs suivants (p.24) : indicateurs subjectifs : présence d'un état de fatigue générale sévère avec 1) perte totale de confiance en soi, résultant en un sentiment d'incompétence et en une aversion profonde envers son travail 2) somatisations multiples sous forme de malaises corporels, de lassitude, d'insomnie, de troubles digestifs ou autres (mais sans maladie organique identifiée) et 3) difficultés de concentration, trouble de jugement et de vigilance, irritabilité et néagtivisme. Indicateurs objectifs : diminuation significative du rendement au travail sur une période de plusieurs mois; cette baisse de performance est obervable 1) aurpès des clients, qui doivent subir des ervices de qualité moindre 2)auprès des superviseurs et employeurs qui observent une baisse du rendement, des retards, des absences, etc. 3) aurpès des collègues, qui constatent une perte d'intérêt chez la personne atteinte. 
Parmi les facteurs d'exclusion qui permettent d'établir un diagnostic différentiel, le comité retient les quatre suivants (p.26) 1) il ne s'agit par d'une imcompétence, puisque l'individu a déjà bien fonctionné dans son emploi et ce durant une période significative, 2) il ne s'agit pas d'une psychopathologie majeure, puisque l'individu n'a pas, dans le passé, présenté les signes d'une psychopatholohie classique (névrose, psychose sévère, trouble de personnalité) et qu'il ne les manifeste pas actuellement; de plus, le fait que la symptomatologie soit grandement améliorée par un retrait du milieu de travail indique qu'il ne s'agit pas d'une psychopathologie classique, 3) il ne s'agit pas, au point de départ, de troubles conjugaux ou familiaux avec répercussions sur le travail, 4) il ne s'agit pas d'une fatigue ou d'un épuisement émotionnel lié à un travail monotone ou à un surcroît de travail, car ces situations ne s'accompagnent pas nécessairement d'un sentiement d'incompétence ni d'une baisse de productivité. 
Le comité conclut à la fin de son rapport qu'il n'est pas nécessaire de créer une nouvelle catégorie nosographique qui porterait le nom de burnout ou de syndrome d'épuisement professionnel et il suggèren plutôt de l'insérer dans la catégorie trouble de l'adaptation avec inhibition au travail. Nonobstant cela, les critères proposés par Bibeau décrivent adéquatement le burnout dysfonctionnel ou clinique de cette étude. 
Le processu de guérison
Bien peu de recherches empiriques ont porté directement sur le processus de guérison de personnes ayant atteint un niveau dysfonctionnel de burnout et aucune ne touche des emplois non professionnels. Dans une étude longitudinale unique, Cherniss (1995) a réalisé une série d'entrevues avec un groupe de professionnels de la relation d'aide ayant ressenti un niveau de stress élevé au cours de leur première année de travail, puis après un internvalle de dix ans. La recherche de Cherniss apporte une contribution importante à la compréhension du burnout. Ses résultats suggèrent que tous ont pu se rétablir de leur burnout, ce qui constitue un excellente nouvelle. Il a identifié quatre facteurs communs chez ses répondants : un changement d'emploi pour obtenir des conditions plus favorables, l'accroissement par l'expérience de l'efficacité personnelle, le développement d'intérêt professionnels spéciaux et une maturité professionnelle quant au choix de carrière. Cependant, cette étude porte encore une fois sur une population de professionnels de la relation d'aide et les participant n'avaient pas tous atteint le niveau critique du burnout. 
Bernier (1993) a efectué des entrevues aurpès d'intervenants et de professionnels provenant de divers milieux pour établir un processus de résolution de la crise du burnout, c'est --à--dire un processus de guérison. Les étapes identifiées dans cette recherche comprennent la reconnaissance du problème, la distanciation des sources de stress, la restauration des capacités, le questionnement des vlauers, l'exploration des possibles, la rupture en tant qu'occasion de transformation et des stratégies d'ajustement. Son étude qualitative est l'un des seuls à explorer le processus de guérison. 
Dans sa recension des écrits sur les internventions thérapeutiques, Kahill (1988) recommande d'améliorer les éléments suivants: le soutien social, les relations interpersonnelles au travail, la variété et la flexibilité des tâches à accomplir et l'équilibre entre le travail et la vie personnelle. Pour sa part, Freudenberger (1983) suggère aux personnes d'accepter le burnout, de se reposer, de fixer leurs priorités, d'affirmer leurs limites, de séparer leurs vies professionnelle et personnelle et de mieux gérer leur temps en réduisant les heures travaillées et en s'occupant davantage d'eux-mêmes.
En conclusion il appret que le processus de guérison des personnes ayant fait l'expérience d'un épuisement professionnel sévère les ayant forcées à s'absenter du travail est mal connu et, qu'en particulier, les connaissances sur les divers niveaux ou catégories d'emploi ne sont pas disponibles. 
Ce qui ressort, c'est que le perfectionnisme est une source de burnout. C'est aspect qu'il faudrait approfondir. Ce qui ressort aussi c'est que la source principale de burnout peut être strictement professionnelle ou bien combinant personnel et professionnel. la difficulté de concilier le travail et la famille est largement élaborée dans la littréature, cependant on n'y fait pas clairement état d'une distinction. Le processus de guérison converge grandement avec les étapes découvertes par Bernier (1993) bien que le découpage et le nom des étapes soient présentés différenemment. La restauration des capacités, le questionnement des valeurs, la rupture comme occasion de transformation et les stratégies d'ajustement suivent en parallèle le processus de guérison observé dans cette étude. Il en va de même pour la majorité des suggestions de Kahill (1988), Freudenberger (1983) et Cherniss (1995). 
Ce s constatations suggèrent que le processus de guérision du burnout peut être généralisé, pour l'essentiel, à toutes les occupations professionnelles, peu importen les stresseurs sociaux, organisationnels ou personnels. Il faut noter cependant une différence quand aux changements de valeurs et de leur mise en action selon le niveau d'emploi. Les cadres et profesionnelles s'orientent davantage vers le couple et la famille; les autres (moins qualifiées) s'orientent d'avantage vers la santé et le soin de soi et doivent apprendre à s'affirmer devant les autres, face aux demandes externes. 
Cette étude apporte un éclairage nouveau sur le burnout en général et sur le processus de guérison en particulier. 
\section {livre : psychologie sociale des organisations, LOUCHE C., 2007. CURUS 2° ed chez Armand Colin, Paris}
Chapitre 1 : pp.1--18
définition du travail : Levy--Leboyer
la qualité des relations interpersonnelles affecte le comportement des salariés (Mayo et Hawthorne)
le style leadership (Lewin)
psychologie industrielle, effet du travail répétitif sur le moral (Myers, 1925)
les concepts fondamentaux de la psychologie sociale (Fischer, 1987)
chapitre 4: le sens du travail: pp.52
Dans les sociétés occidentales le travail avait progressivement constitué une valeur forte. En effet, les grands courants de pensée (le christianisme, le marxisme, le libéralisme) dans lesquels beignent ou ont baigné ces sociétés ont des conceptions proches et valorisées du travail (Meda, 1995).
Il fait les spécificités de l'homme et permet de le définir
Il permet 'lintégration sociale, développe le sentiment d'appartenance à la société
Il constitue un facteur de développement personnel et de réalisation de soi.
mauvaises critiques : Levy--Leboyer (1984) et Rousselet (1974)
chapitre 6 : la motivation au travail : pp71--83
La crise de la motivation Levy--Leboyer (1984)
Les caractéristiques essentielles de la motivation : elle suscite le déclenchement de comportements, les dirige vers certains buts avec une certaine intensité. Enfin, elle amène à persister jusqu'à l'atteinte des objectifs. Les modèles théoriques rendant compte de la motivation sont multiples.
chapitre 7: l'implication dans le travail, l'engagement organisationnel et la satisfaction : pp 84
Salah et Hosek (1976) quatre significations différentes: 
-- la mesure dans laquelle le travail est central pour l'individu en lui permettant de satisfaire des besoins valorisés (Dubin, 1956)
-- la mesure dans laquelle une particpation active permet la réalisation de soi (Guin, Veroff, Field, 1960)
-- la mesure dans laquelle les performances au travail affectent l'estime de soi (Lodhal et Kejner, 1965; Freuch et Kahm, 1962)
-- la consistance entre les performances et les conceptions de soi (Vroom, 1962)
la relaion entre l'implication au travail et des variables personnelles et situationnelles
Brown classe en premier les variables personnelles et situationnelles. C'est une classification à rôle causal (1) et à rôle non causal (2)
1) rôle causal : 
variables de personnalités (construites pendant l'éducation) : locus of control (l'estime de soi et du besoin de développement personnel)
caractéristiques de l'emploi (sentiement de responsabilité: un travail chargé de sens et un feedback sur les performances)
relations à la hierarchie (considération, participation et communication)
2) rôle non causal : essentiellement des variables démographiques (âge, niveau d'éducation, sexe...)
La satisfaction au travail 
définition : la satisfaction au travail est un état émotionnel résultant de la relation persque entre ce que l'on veut obtenir de son travail et ce qu'il nous apporte (Locke, 1969)
Friedlander, 1963 : trois facteurs source de satisfaction:
1) l'environnement social et technique
2) les composantes internes du travail et la réalisation de soi
3) reconnaissance par l'avancement (responsabilité...)
chapitre 9: santé psychologique au travail : pp104
Selye, 1956 : concept de stress introduit pour rendre compte des réponses biologiques physiologiques ou comportementales de l'organismes à un agent pathogène externe ou interne. Il désigne à la fois les facteurs qui placent l'individu en position difficile
Lanery et Ponelle, 2004 : le stress est un état perçu comme négatif par un groupe de travailleurs qui s'accompagne de plaintes ou disfonctionnements au niveau physique, psychique et ou social et qui est la conséquence du fait que les travailleurs ne sont pas en mesure de répondre aux exigences et attentes qui leur sont posées par leur situation de travail.
Les approches du stress sont nombreuses et ont fait l'objet de revue de question (Rolland, 1999; Rascle, 2004; Van de Leemput, 2005).
Orientation mécaniste : les facteurs environnementaux responsables du développement de tensio au travail :
-- contenu du travail (surcharge, complexité, pression temporelles, rythme...)
-- organisation du travail (conflit...)
-- relations sociales
-- environnement physique 
Orientation interactionnelle
modèle Job-demand control (Krasek, 1979) pour comprendre les tensions au travail deux variables peuvent être utilisées: 
1) exigence de la situation au travail (contraintes psychologiques qui pèsent sur le salarié, pressions temporelles, charge de travail, conflit de rôle...)
2) lattitude de décision: renvoie à l'autonomie dont dispose le salarié ainsi qu'aux capacités de réalisation de soi
discussion des variables du modèle, intégration d'une troisième dimension : support social (collègues, hierarchie)
l'environnement de travail, situation de l'individu : considéré comme figés
3) conception transactionnelle
Lazarus et Folkman, 1984 : le stress est une relation particulière entre la personne et l'environnement, relation qui est évaluée par l'individu comme excédant ses ressources et menaçant son bien être.
4) gesion du stress
le travail dans un contexte de stress s'accompagne, pour les salariés, de risques au niveau physique et psychologique

\section {livre : psychologie du travail comprendre et analyser le comportement de l'homme au travail. Théories et applications. Guillevic C., 2002, ed : Nathan universités, p 255}
chapitre 7: performance et rendrement du système socioéconomique : pp 167--191
L'activité déployée par l'opérateur pour répondre aux exigences du système se traduit par une charge de travail.
Spérandio, 1988 : il est classique de distinguer pour la charge de travail les contraintes qui s'exercent sur l'opérateur et les astreintes qui prenrésentent un coût physique et mental pour l'opérateur.
Evaluation de la charge de travail physique = mesure physiologique de l'effort : mesures directes sur le métabolisme (échanges respiratoire, déposne énergétique, rythme cardiaque...) et mesure indirectes (électrocardiographie, éléctroencéphalogramme...)








\end{document}
BIBLIOGRAPHIE

A
American Psychiatric Association (1994). DSM-IV
ANTONOVSKY A. (1987). Unraveling the mystery of health. How people manage stress and stay well. San Francisco: Jossey-Bass

B
BAKKER A.B., DEMEROUTI E., DE BOER E. et SCHAUFELI W.B. (2003). Job demands and job ressources as perdictors of abscence duration and frequency, Journal of vocational behaviour, 62 (2), 341-356
BEEHR T.A. (1995). Psychological stress in the workplace. New York : Routledge 
BENOIT N. (2005). S’épuiser et en guérir : analyse de deux trajectoires selon le niveau d’emploi, Revue internationales sur le travail et la société, 3 (2), 835-865
BERNIER D.(1993). La crise du burnout. Montréal, Canada :Stanké
BERNNINKMEIJER V., VAN YPEREN N.W. et BUUNK B.P. (2001). I am a better teacher, but other are doing worse : burnout and perceptions of superiority among teachers, Social Psychology of Education, 43 (3-4), 259-274
BIBEAU G., DUSSAULT G., LAROUCHE L. M., LIPPEL K., SAUCIER J.F., VEZINA M. et al. (1989). Certains aspects culturels diagnostiques et juridiques du burnout, pistes et repères opérationnels, Montréal, GIRAME, Université de Montréal
BOUDOUKHA A.H. (2006). Etude conjointe du burnout et des troubles de stress post-traumatique dans une population à risques, dans Cas des professionnels en milieu carcéral, Lille 3 Charles de Gaulle, Villeneuve d’Ascq
BREHM J.W. (1966). A psychological theory of reactance. New York : academic press


C
CHADWICK-JONES J.K., NICHOLSON N. et BROWN C. (1982). Social psychology of absenteeism. New York : Praeger
CHERNISS C.(1995). Beyond burnout: helping teachers, nurses, therapists and lawyers recover from stress and disillusionment. New York: Rouledge)CHERNISS C. (1980). Staff burnout. Job stress in the human services, Beverly Hills, Sage
COHEN A. (1991). Carrer stage as a moderator of the relationships between organizational commitment and its outcomes : a meta-analysis. Journal of occupational psychology, 64, 253-268
COTE L, EDWARDS H. et BENOIT N. (2005). S’épuiser et en guérir : analyse de deux trajectoires selon le niveau d’emploi, revue internationale sur le travail et la société, 3 (2), 835-865). Le repérage du trouble de burnout ou de burnout pathologique pose différents 
COTE L., EDWARDS H., SCHAUFELI W.B. et ENZMAN D. (1998). The burnout companion to study and practice, Palo Alto, Taylor and Francis)
CORDES C.L. et DOUGHERTY T.W. (1993). A review and an integration of research on job burnout, Academy of management review, 18 (4), 621-656). 
CRESSWELL S.L. et EKLUND R.C. (2004). The athlete burnout syndrome : possible early signs, Journal of sciences and medecine in sport, 7 (4), 481-487) 

D
DEMEROUTI E., BAKKER A.B., NACHREINER F.,  SCHAUFELI W.B. (2001). The job demands-ressources mode of burnout. Journal of applied psychology, 86, 499-512
DE VENTE W., OLFF M., VAM AMSTERDAM J.G.C., KAMPHUIS J.H. et EMMELKAMP P. (2003). Physiological differences between treatement for posttraumatic stress disorder, Behaviour Therapy, 26, 487-499
DION G. et TESSIER R. (1994). Validation de la traduction de l’inventaire d’épuisement professionnel de Maslach et Jackson, Revue canadienne des sciences du comportement, 26, 210-227
DWYER D.J. et GANSTER D.C. (1991). The effects of job demands and control on employee attendance and satisfaction. Journal of organizational behaviour, 12, 595-608

E
EDLEWICH J. et BRODSKY A. (1980). Burnout : stages of disillusionment in the helping professions, New York, Human sciences press
ERICKSON R.J., NICHOLS L. et RITTER C. (2000). Family influences on absenteeism : testing an expanded process model. Journal of vocational behaviour, 57, 246-272

F
FARELL D. et STAMM C.L. (1988). Meta-analysis of the correlates of employee absence. Human relations, 41, 211-227
FERNET C, GUAY F. et SENECAL C. (2004). Adjusting to job demands : the role of work self-determination and job control in predicting burnout, dans journal of vocational behaviour, 65 (1), 39-56
Freudenberger H.J.(1983). Hazards of psychotherapeutic practice. Psychitherapy in private practice, 1, 83-89
FRENDENBERGER H. (1977). Burn-out : the organizational menace, Training and developpement journal, 31 (7), 26-27

G
GOLEMBIEWSKI R.T., BOURDEAU R.A., MUNZENRIDER R.F. et HUAPING L. (1996). Global burnout : a worldwide pandemic explored by the phase model. London: JAI Press
GOLOMBIEWSKI R.T., BOURDEAU R.A. et LUO H. (1994). Global burnout : a worldwide pandemic explored by the model phase, Greenwich, JAI Press 
GREENGLASS E.R., RONALD J.B. et MOORE K.A. (2003). Reaction to increased workload : effects on professionnal efficacy of nurses, Applied Psychology : an international review, 52 (4), 580-597 
GUERITAULT-CHALVIN V. et COOPER C. (2004). Mieux comprendre le burnout professionnel et les nouvelles stratégies de prévention : un compte rendu de la littérature, Journal de thérapie comportementale et cognitive, 14 (2), 59-70
GELLATLY I.R. (1995). Individual and group determinants of employee absenteeism : test of causal model. Journal of organizational behaviour, 16, 469-485


H

HACKMAN J.R. et OLDHAM G.R. (1980). Methodological issues in the use of absence data. Journal of applied psychology, 66, 574-581 
HACKMAN J.R. et OLDHAM G.R. (1976). Motivation through the desing of work : test of a theorty. Organizational behaviour and human performance, 16, 250-279
HAUTEKEEETE M. (2001). Comment définir un ensemble de concepts complexes : stress, adaptation et anxiété, in P. GRAZIANI, M. HAUTEKEETE, S. RUSINEK et D. SERVANT (ed.), Stress, anxiété et troubles de l’adaptation, Paris, Acanthe/Masson, p.1-13
HENSING G., ALEXANDERSON K., ALLEBACK P. et BJURULF P. (1998). How to measure sickness absence ? literature review and suggestion of five basic measures. Scandinavian Journal of social medecin, 26, 133-144

I
J
JAMAL M. (1984). Job stress and job performance controversy : an empirical assesment. Organizational behaviour and human performance, 33, 1-21
JOHNS G. (1997). Contemporary research on absence from work : correlates, causes and consequences. International review of industrial and organizational psychology, 12, 115-173

K
KAHILL S.(1988). Interventions for burnout in the helping professions : a review of the empirical evidence. Canadian journal of consuelling, 22, 161-169
KAHILL S. (1988). Symptoms of professional burnout : a review of the empirical evidence, Canadian Psychology, 76, 523-537
KOESKE G.F. et KOESKE R.D. (1989). Construt validity of the maslach burnout inventory : a critical review and reconceptualization. Journal of applied behavioral science, 25, 131-144
KOP N., EUWEMA M. et SCHAULEFI W.B. (2003). Burnout, job stress and violent behaviour among Dutch police officiers, Work and stress, 13 (4), 326-340
KULIK L. et RAYYAN F. (2003). Spousal relation and well-being : a comparative analysis of Jewish and Arab dual-earner families in Israel, Journal of community psychology, 31 (1), 57-73

L
LAFLEUR J. (1999). Le burnout : questions et réponses. Outremont, Québec : les Editions Logiques
LAWSON D.A. et O’BRIEN R.M. (1994). Behavioral and self-report measures of staff burnout in development disabilities. Journal of organizational behavior management, 14, 37-54
LAZARUS R.S. et FOLKMAN S. (1984). Stress, appraisal and coping, New York, Springer 
LEE E.T. et ASHFORTH B.E. (1996). A meta-analytic examination of the correlates of the three dimensions of job burnout. Journal of applied psychology, 81, 123-133
LEBIGOT F. et LAFONT B. (1985). Psychologie de l’épuisement professionnel, Annales médico-psychologiques, 143 (8), 769-775
LEITER M.P. et SCHAUFELI W.B. (1996). Consistency of the burnout construct across occupations, Anxiety, stress and coping, 9, 229-243
LIDVAN-GIRAULT N., (1989). Burnout : émergence et stratégie d’adaptation. Le cas de la médecine d’urgence, Paris, Université Paris 5 René Descartes
LOUREL M., GANA K., PRUD’HOMME V. et CERCLE A. (2004). Le burnout chez les personnels des maisons d’arrêt : test du modèle « demande-contrôle » de Karasek, L’encéphale, XXX, 557-563

M
MASLACH C. (2001). What we learned about burnout and health ? Psychology and Health, 16, 607-611
MASLACH C. (1993). Bunout : a multidimensional perspective. In W.B. SCHAUFELI, C. MASLACH et T. MAREK (Eds), professional burnout: recent developements in theory and research (pp. 19-33). Washington, DC: Taylor et Francis
MASLACH C. (1982). Understanding burnout : definitional issues in analysis a complex phenomenon, in W.S. PAINE (ed.), Job stress and burnout : research, theory and intervention perspectives, Beverly Hills, CA, Sage Publication, p. 29-40
MASLACH C. et LEITER M.P. (1997). The truth about burnout : how organizations cause personal stress and what to do about it. San Francisco, CA: Jossey-Bass Publishers
MASLACH C., JACKSON S.E. et LEITER M.P. (1996). Maslach burnout inventory. Manual (3rd ed). Palo Alto, CA : consulting psychologists press 
MASLACH C. et JACKSON S.E. (1986). The measurement of experienced burnout, Journal of occupational behaviour, 2, 99-113
MASLACH C. et SHAUFELI W.B. (1993). Historical and conceptual development of burnout. In W.B. SCHAUFELI et T. MAREK (Eds), Professional burnout : recent developments in theory and research (pp;1-19). Washington, DC: Taylor et Francis
MASLACH C, SCHAUFELI W.B. et LEITER M.P. (2001). Job burnout, Annu.Rev. Psychol., 52, 397-422). MASLACH C. et SCHAUFELI W.B. (1993). Historical and conceptual developpement of burnout, in W.B. SCHAUFELI, C. MASLACH et T. MARCK (ed.), Professional burnout : recent developments in the theory and research, Washington, Taylor et Françis, p.1-16
MASLACH C., SHAUFELI W.B. et LEITER M.P. (2001). Job burnout, Annu. Rev. Psychol., 52, 397-422
MATHIEU J.E. et KOHLER S.S. (1990). A test of the interactive effects of organizational commitment and job involvement on various types of absence. Journal of vocational behaviour, 36, 33-44
MAURANGES A. et CANOUÏ P. (2001). Le syndrome d’épuisement professionnel des soignants : de l’analyse du burnout aux réponses, Paris, DUNOD).
MC MANUS I.C., WINDER B.C. et GORDON D. (2002). Dissociation and posttraumatic stress disorder : two prospective studies of motor vehicle accident survivors. British journal of psychiatry, 180, 363-368). 
MELAMED S., BEN-AVI I., LUZ J. et GREEN M.S. (1995). Ojective and subjective work monotony : effects of job satisfaction, psychological distress, and absenteeisme in blue-collar workers. Journal of applied pasychology, 80, 29-42
MEYER J.P. et ALLEN N.J. (1991). A three-component conceptualization of organizational commitment. Human resource management review, 1, 61-98

N
NICHOLSON N. (1993). Absence-there and back again. Journal of organizational behaviour, 14, 288-290
O
OSCA A., GONZALEZ-CARMINO G., BARDERA P. et PEIRO J.M. (2003). Estès de rol y su influencia sobre bienestar fisico y psychoquico en soldas professionals, Psicothema, 15 (1), 12-17), 

P
PINES A., ARONSON E. et KAFRY D. (1981). Burnout : from tedium to personnal growth, New York, Free Press) 
PRICE L. et SPENCE S.H. (1994). Burnout symptoms amongst drug and alcohol service employees : gender differences in the interaction between work and home stressors. Anxiety, stress and coping, 7, 67-84

Q

R

S
SAGIE A. (1998). Employee absenteeism, organizational commitment, and job satisfaction : another look. Journal of vocational behaviour, 52, 156-171
SCARFONE D. (1985). Le syndrome de l’épuisement professionnel (burnout) : y aurait-il de la fumée sans feu ?, Annales médico-psychologiques, 143 (8), 110-122
SCHAT A.C.H. et KELLOWAY K.E. (2003). Reducing the adverse consequences of workplace agression and violence : the buffering effects of organizational support, journal of occupational health psychology, 8 (2), 110-122
SCHAUFELI W.B., BAKKER A.B., HOOGDUIN K., SCHAAP C. et KLADLER A. (2001). On the clinical validity of the Maslach burnout inventory and the burnout measure, Psychology and health, 16, 565-582 
SCHAUFELI W.B. et ENZMAN D. (1998). The burnout companion to study and practice, Palo Alto, Taylor and Françis
SCHAUFELI W.B.,  MASLACH C. et MAREK T. (1993). The future of burnout. In W.B. SCHAUFELI, C. MASLACH et T. MAREK (Eds), Professional burnout: Recente developments in theory and research (pp. 253-260). Washington, DC: Taylor et Francis
SHIROM A. (1989). Burnout in work organization. In C.L. Cooper and I. Robertson (Eds.), intenrational review of industrial and organizational psychology (pp. 25-48). New York : Wiley 
SHIROM A., MELAMED S., TOKER S., BERLINER S. et SHAPIRA I. (2005). Burnout, mental and physical health : a review of the evidence and proposed explanory model, dans International Review of Industrial and Organaizational Psychology, 20, 269-309

T
TRUCHOT D. (2004). Epuisement professionnel et burnout : concepts, modèles, interventions, Paris, DUNOD).
TRUCHOT D. et BADRE D. (2004). Agressions au travail et burnout chez les travailleurs sociaux : influence de l’auto-efficacité et du pouvoir organisationnel, dans stress et trauma, 4 (3), 187-194). 
TRUCHOT D. et BADRE D. (2003). Equity and the burnout process : the role of helping models, dans Revue internationale de psychologie sociale, 16 (4), 5-24). 

V
VARTIA M. et HYYTI J. (2002). Gender diffrences in workplace bullying among prison officiers, dans European journal of work and organzational psychology, 11 (1), 113-126). 
VENIGA R.L. et SPRADLEY J.P. (1981). The work/stress connection : how to cope with job burnout, Boston, Little, Brown)

W
WEBER A. et JAEKEL-REINHARD A. (2000). Burnout syndrome : a disease of modern societies ?, dans Occupational medecine, 50 (7), 512-517
WESTMAN M. et ELTZION D. (1995). Crossover of stress, strain and ressources from spouse to another, dans Journal of organizational behaviour, 16 (2), 169-181
WOPLIN J., BURKE R.J. et GREENGLASS E.R. (1991). Is job satisfaction as antecedent or a consequence of psychological burnout ? dans Human Relation, 44, 193-209

X
Y
YOUNGBLOOD S.A. (1984). Work, nonwork, and withdrawal. Journal of applied psychology, 69, 106-117

Z



